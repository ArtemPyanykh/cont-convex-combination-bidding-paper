\documentclass[12pt]{extarticle}

\usepackage{geometry}
\geometry{a4paper}

\usepackage{pscyr}

\usepackage[T2A]{fontenc}
\usepackage[utf8]{inputenc}
\usepackage[english,russian]{babel}

\usepackage{amsmath}
\usepackage{amssymb}

\usepackage{color}

\begin{document}

\begin{center}
{\Large Ответ рецензенту}
\end{center}

В соответствии с Вашими замечаниями статья доработана. Ниже приведен подробный
разбор изменений. Благодарю Вас за внимательное прочтение статьи и ценные
замечания, позволившие улучшить ее текст.

\begin{flushright}
С уважением, Пьяных А.И.
\end{flushright}


\begin{itemize}
\item%

  \textcolor{blue}{%
    Напомню, что цель работы Де Мейера и Салей, что видно из ее названия
    \textit{On the Strategic Origin of Brownian Motion in Finance}, на примере
    упрощенной модели биржевых торгов продемонстрировать возможность
    стратегического происхождения броуновского движения в финансах. Де Мейер и
    Салей устанавливают, что при $n$ стремящемся к бесконечности значения
    $n$-шаговых игр растут со скоростью $\sqrt{n}$, получают асимптотику
    случайной последовательности цен сделок и демонстрируют наличие в этой
    асимптотике винеровской компоненты.\\
    Артему Игоревичу Пьяных следовало бы завершить работу, получив асимптотику
    для рассматриваемого им случая $\beta \neq 1$, и сравнить ее с упомянутой
    выше асимптотикой при $\beta = 1$.}
  
  Действительно, в силу того, что содержательная часть статьи была посвящена
  решению конечно-шаговых игр, вопросы связанные с асимптотикой не были
  освещены.
  
  Один из результатов рецензируемой работы, состоит в том, что хотя стратегии
  игроков зависят от $\beta$, значение игры от него не зависит и совпадает со
  значением в работе De Meyer, Saley (2002). Отсюда содержание следствия 4,
  теоремы 12 и леммы 9 из De Meyer, Saley (2002) об асимптотике значений игры
  $V_n(P)$, динамике апостериорных вероятностей $P_n$ и последовательности
  ставок $p_{1,n}$ справедливо и для данной модели без каких-либо изменений.
  
  В этом смысле результат согласован с работой De Meyer (2010), в которой
  показано, что асимптотика последовательности ожидаемых ликвидных цен акции не
  зависит от конкретного вида механизма торгов. Хотя формально, чтобы
  воспользоваться данным фактом для механизма из рецензируемой работы,
  необходимо распространить соответствующие результаты на случай модели с
  произвольным априорным распределением цены акции с конечным мат. ожиданием.
  
\item%
  \textcolor{blue}{%
    Далее, Бернаром Де Мейером в работе De Meyer B. (2010) для моделей с весьма
    общим торговым механизмом получена асимптотика последовательности цен
    сделок. Но поскольку работа De Meyer (2010) не касается решений
    конечно-шаговых игр, результат рецензируемой работы является новым. Тем не
    менее упоминание работы De Meyer B. (2010) необходимо.}

  Как было замечено, результат работы De Meyer (2010) не касается
  конечно-шаговых игр. Кроме того, одно из условий на механизм торгов, в рамках
  которых автором получены результаты, является существование значения
  $n$-шаговых игр при любом априорном распределении цены акции с конечным мат.
  ожиданием.

  Обсуждение результатов работы De Meyer (2010) внесено в статью.
  
\item%
  \textcolor{blue}{%
    На странице 2 автор пишет: ``В обеих работах использовался одинаковый
    механизм проведения сделки, \dots''. Этой фразе предшествует упоминание
    ЧЕТЫРЕХ работ [1-4]. Мне понятно, что автор имел ввиду работы [1] и [2], но,
    кстати, в работах [3] и [4] используется тот же механизм проведения сделки,
    что и в работах [1] и [2].}

  Действительно, во всех вышеупомянутых во введении работах был использован один
  и тот же механизм проведения сделки. Соответствующая фраза в статье
  исправлена.
\end{itemize}
\end{document}
